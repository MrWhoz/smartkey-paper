\chapter{Tổng kết}

% **************************** Define Graphics Path **************************
\ifpdf
    \graphicspath{{Chapter4/Figs/Raster/}{Chapter4/Figs/PDF/}{Chapter4/Figs/}}
\else
    \graphicspath{{Chapter4/Figs/Vector/}{Chapter4/Figs/}}
\fi
\section{Thành quả đạt được và khó khăn}
\subsection{Thành quả đạt được}
Sau thời gian 7 tháng thực hiện đề tài NCKH, nhóm đã đạt được các thành quả:

• Tìm hiểu được về công nghệ Bluetooth Low Energy và các thiết bị phần cứng hỗ trợ.

• Kết nối Bluetooth với thiết bị di động theo chuẩn Bluetooth BLE 4.0.

• Đạt được các chức năng trên thiết bị Smart Keyring đã được đề ra.

• Khả năng hoạt động liên tục đến 430 giờ ở chế độ idle.

• Hiện thực ứng dụng thiết bị di động trên Android.

\subsection{Khó khăn}

Bên cạnh những thành quả đạt được, nhóm đã vấp nhiều khó khăn:

• Tốn nhiều thời gian cho phương pháp tiếp cận ban đầu không khả thi vì gặp trở ngại trong cả phần cứng lẫn phần mềm trong việc phát triển (đã được nhắc đến ở mục \ref{dev})

• Không có đủ thời gian để thực nghiệm đầy đủ hơn.

• Không đủ thiết bị di động có chuẩn BLE cho tất cả thành viên trong nhóm làm việc hiện thực và thực nghiệm gặp nhiều hạn chế.

• Lựa thiết bị thiết bị phần cứng ở ở Việt Nam tại thời điểm tìm hiểu và phát triển đề tài NCKH.

• Ứng dụng di động trên Android chưa phát triển hoàn thiện và mất nhiều thời gian tìm hiểu lập trình trên Android cho người mới bắt đầu.

\section{Tính ứng dụng thực tiễn của sản phẩm}

Theo đánh giá về tính ứng dụng thực tiễn được dựa theo mục \ref{result}, ta rút ra kết luận:

• Khoảng cách hoạt động chấp nhận được.

• Các chức năng hoạt động tốt và có lợi ích trong cuộc sống hằng ngày.

• Ứng dụng di động trên Android chưa hỗ trợ chế độ chạy nền nên hơi bất tiện cho người dùng trong việc sử dụng hằng ngày.

• Thời gian hoạt động kém hiệu quả, chưa sử dụng tốt được trong cuộc sống hằng ngày.

\section{Hướng phát triển}

• Tìm hiểu các linh kiện điện tử khác hỗ trợ lập trình SoC BLE để tối ưu hóa hiệu năng hoạt động cũng như kích thước của sản phẩm.

• Thiết kế ngoại hình bên ngoài sản phẩm để ứng dụng thực tế hơn.

• Phát triển ứng dụng Android hỗ trợ chế độ chạy ngầm.

• Có thể tùy chỉnh được khoảng cách kết nối.
