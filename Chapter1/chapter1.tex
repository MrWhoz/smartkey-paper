%*******************************************************************************
%*********************************** First Chapter *****************************
%*******************************************************************************

\chapter{Tổng quan}  %Title of the First Chapter

\ifpdf
    \graphicspath{{Chapter1/Figs/Raster/}{Chapter1/Figs/PDF/}{Chapter1/Figs/}}
\else
    \graphicspath{{Chapter1/Figs/Vector/}{Chapter1/Figs/}}
\fi


%********************************** %First Section  **************************************
\section{Mở đầu} %Section - 1.1 
\subsection{Giới thiệu}
Ngày nay, xã hội phát triển mạnh mẽ, kỹ thuật ngày càng hiện đại nên nhu cầu về trao đổi thông tin, giải trí, nhu cầu về điều khiển thiết bị từ xa,…ngày càng cao. Và những hệ thống dây cáp phức tạp lại không thể đáp ứng tốt nhu cầu này, nhất là ở những khu vực chật hẹp, những nơi xa xôi, trên các phương tiện vận chuyển,…Vì thế công nghệ không dây đã ra đời và đang phát triển mạnh mẽ, tạo rất nhiều thuận lợi cho con người trong đời sống hằng ngày. Kỹ thuật không dây phục vụ rất nhiều nhu cầu khác nhau của con người, từ nhu cầu làm việc, học tập đến các nhu cầu giải trí như chơi game, xem phim, nghe nhạc, v.v…Với các nhu cầu đa dạng và phức tạp đó, kỹ thuật không dây đã đưa ra nhiều chuẩn với các đặc điểm kỹ thuật khác nhau để có thể phù hợp với từng nhu cầu, mục đích và khả năng của người sử dụng như IrDA, WLAN với chuẩn 802.11, ZigBee, OpenAir, UWB, Bluetooth,…

Mỗi chuẩn kỹ thuật đều có những ưu, khuyết điểm riêng của nó, và Bluetooth đang dần nổi lên là kỹ thuật không dây tầm ngắn có nhiều ưu điểm, rất thuận lợi cho những thiết bị di động. Với một tổ chức nghiên cứu đông đảo, hiện đại và số lượng nhà sản xuất hỗ trợ kỹ thuật Bluetooth vào sản phẩm của họ ngày càng tăng, Bluetooth đang dần lan rộng ra khắp thế giới, xâm nhập vào mọi lĩnh vực của thiết bị điện tử và trong tương lai mọi thiết bị điện tử đều có thể được hỗ trợ kỹ thuật này.
\subsection{Mục tiêu - Phạm vi - Đối tượng nghiên cứu}

Xuất phát từ các lý do trình bày ở trên, chúng tôi đã thực hiện đề tài “Thiết kế sản phẩm móc khóa thông minh (Smart Keyring) dựa trên nền tảng công nghệ Bluetooth Low Energy (BLE)”. Mục tiêu của đề tài là:

- Tìm hiểu về công nghệ Bluetooth

- Kế thiết bị “Móc khóa thông minh - SmartKeyring” sử dụng công nghệ

Bluetooth và kết nối với ứng dụng Android trên điện thoại.

\nomenclature[z-cif]{$CIF$}{Cauchy's Integral Formula}                                % first letter Z is for Acronyms 
\nomenclature[a-F]{$F$}{complex function}                                                   % first letter A is for Roman symbols
\nomenclature[g-p]{$\pi$}{ $\simeq 3.14\ldots$}                                             % first letter G is for Greek Symbols
\nomenclature[g-i]{$\iota$}{unit imaginary number $\sqrt{-1}$}                      % first letter G is for Greek Symbols
\nomenclature[g-g]{$\gamma$}{a simply closed curve on a complex plane}  % first letter G is for Greek Symbols
\nomenclature[x-i]{$\oint_\gamma$}{integration around a curve $\gamma$} % first letter X is for Other Symbols
\nomenclature[r-j]{$j$}{superscript index}                                                       % first letter R is for superscripts
\nomenclature[s-0]{$0$}{subscript index}                                                        % first letter S is for subscripts


%********************************** %Second Section  *************************************
\section{Tổng quan về công nghệ Bluetooth} %Section - 1.2

\subsection{Khái niệm Bluetooth}
Bluetooth là công nghệ không dây cho phép các thiết bị điện, điện tử giao tiếp với nhau trong khoảng cách ngắn, bằng sóng vô tuyến qua băng tần chung ISM (Industrial, Scientific, Medical) trong dãy tầng 2.40- 2.48 GHz. Đây là dãy băng tầng không cần đăng ký được dành riêng để dùng cho các thiết bị không dây trong công nghiệp, khoa học, y tế.

Bluetooth được thiết kế nhằm mục đích thay thế dây cable giữa máy tính và các thiết bị truyền thông cá nhân, kết nối vô tuyến giữa các thiết bị điện tử lại với nhau một cách thuận lợi với giá thành rẻ. Khi được kích hoạt, Bluetooth có thể tự động định vị những thiết bị khác có chung công nghệ trong vùng xung quanh và bắt đầu kết nối với chúng. Nó được định hướng sử dụng cho việc truyền dữ liệu lẫn tiếng nói.

\subsection{Quá trình phát triển}
% TODO: nguồn wiki link chú thích bla bla bla
Đặc tả Bluetooth được phát triển đầu tiên bởi Ericsson (hiện nay là Sony Ericsson và Ericsson Mobile Platforms), và sau đó được chuẩn hoá bởi Bluetooth Special Interest Group (SIG). Chuẩn được phát hành vào ngày 20 tháng 5 năm 1999. Ngày nay được công nhận bởi hơn 1800 công ty trên toàn thế giới. Được thành lập đầu tiên bởi Sony Ericsson, IBM, Intel, Toshiba và Nokia, sau đó cùng có sự tham gia của nhiều công ty khác với tư cách cộng tác hay hỗ trợ. Bluetooth có chuẩn là IEEE 802.15.1.

\textbf{Các phiên bản Bluetooth:}

% TODO: nguồn thegioididong.com
1.    Bluetooth 1.0: Là phiên bản đầu tiên của chuẩn kết nối Bluetooth được đưa vào sử dụng với tốc độ truyền tải dữ liệu là 1Mbs, tuy nhiên thực tế tốc độ của phiên bản này chỉ đạt được mức 720kbs.

2.    Bluetooth 2.0 + ERD: Phiên bản nâng cấp sau Bluetooth 1.0 được nâng cấp tốc độ truyền tải lên 2.1 Mbs cùng với chế độ truyền tải mới ERD (enhanced data rate). Phiên bản 2.1 được nâng cấp về tốc độ truyền tải nhưng lại hạn chế trên thiết bị sử dụng do ERD chỉ là chế độ tùy chọn, một số nhà sản xuất đã không đưa chế độ này vào sản phẩm của mình để giảm chi phí sản xuất.

3.    Bluetooth 2.1+ ERD: Được nâng cấp từ Bluetooth 2.0 vào năm 2007 với thay đổi quan trọng như hiệu năng cao hơn, giảm điện năng tiêu thụ. Phiên bản này được sử dụng trên các thiết bị như điện thoại di độn, laptop, tai nghe ….. Tuy nhiên, Bluetooth 2.1 vẫn chưa cho người dùng truyền tải các tập tin có dung lượng lớn.

4.    Bluetooth 3.0 + HS: Năm 2009 buetooth 3.0 ra đời với thay đổi lớn về tốc độ truyền tải, đạt 24Mbps ở phiên bản này các thiết bị có thể tương tác dễ dàng với nhau hơn, có thể tự dò tìm các thiết bị ở gần.

5.    Bluetooth 4.0: Là sự kết hợp của các đời Bluetooth trước đó với nhau. Bluetooth 4.0 đạt tốc độ truyền tải lên đến 25Mbps, dễ dàng ghép đôi các thiết bị với nhau, hiệu năng tiêu thụ thấp. Đây là chuẩn Bluetooth được sử dụng trên hầu hết các thiết bị hiện nay.

6.    Bluetooth 4.1 và 4.2: Là phiên bản ra đời đầu năm 2014 với nhiều cải tiến vượt bậc so với Bluetooth 4.0 như khả năng điều chống chồng chéo tín hiệu, kết nối thực sự thông minh và khả năng truyền dữ liệu độc lập mà không cần phụ thuôc vào trung tâm điều khiển. Phiên bản 4.2 được phát triển có khả năng truyền tải cao và bảo mật hơn, nhưng quan trọng hơn cả là cho phép các vi xử lý sử dụng chuẩn giao thức Ipv6 để truy cập trực tiếp vào internet.

7.	Bluetooth 5.0: theo dự kiến sẽ bắt đầu xuất hiện trên các thiết bị thương mại vào cuối 2016 nay hoặc đầu năm 2017 (Q1). Bluetooth 5.0 có tầm phủ sóng tăng lên gấp 4 lần so với Bluetooth 4.2 hiện nay, còn tốc độ truyền dữ liệu thì tăng lên cao nhất là 2 lần. Việc mở rộng khả năng phủ sóng của Bluetooth sẽ giúp các thiết bị Internet of Things sẽ có thể giao tiếp với nhau cũng như với trạm điều khiển một cách dễ dàng hơn, vượt qua bức tường của một căn nhà bình thường, trong khi lại tăng tốc thu thập và truyền dữ liệu. Chuẩn Bluetooth mới cũng sẽ giúp các beacon và giải pháp nhận diện địa điểm trở nên thông minh, chính xác và phản hồi nhanh hơn với sự hiện diện của người dùng.

% theo http://www.pcmag.com/news/345316/bluetooth-5-0-to-quadruple-range-double-speed
%********************************** % Third Section  *************************************
\section{Tại sao chọn Bluetooth Low Energy (BLE)?}  %Section - 1.3 
\label{section1.3}
\begin{center}
\begin{table}
	\begin{tabular}{ |c|m{2cm}|m{2cm}|m{2cm}|m{2cm}| } 
		\hline
		  & \textbf{Bluetooth} & \textbf{BLE} & \textbf{Wifi} & \textbf{Zigbee} \\ 
		  \hline
		\textbf{Radio Frequency} &	2.4G &	2.4G &	2.4G &	2.4G \\ 
		\hline
		\textbf{Distance Range} &	10m	&>60m &	30m	& 10-100m \\ 
		\hline
		\textbf{Air Datarate} &	1-3Mbps &	1Mbps &	54Mbps &	250kbps \\
		\hline
		\textbf{Application Throughput} &	0.7-2.1Mbps &	305kbps &Depend &120kbps\\
		\hline
		\textbf{Security} &	64bit, 128bit &	128-bit & AES	SSID, WEP&	128-bit AES \\
		\hline
	\textbf{Power consumption}&	Low	&Very Low&	High&	Low \\
	\hline
	\textbf{Certification Body}&	Bluetooth SIG&	Bluetooth SIG&	IEEE802.11&	IEEE802.15.4 \\
	\hline
	\textbf{Network topology} &	Point-to-Point Scatternet&	Point-to-Point Star&	Point-to-Hub& 		Mesh, Ad-hoc\\
		\hline
		\end{tabular}
	
	\caption {Should be a caption}
	\label{table:1.3.1}
\end{table}
	
\end{center}


\nomenclature[z-DEM]{DEM}{Discrete Element Method}
\nomenclature[z-FEM]{FEM}{Finite Element Method}
\nomenclature[z-PFEM]{PFEM}{Particle Finite Element Method}
\nomenclature[z-FVM]{FVM}{Finite Volume Method}
\nomenclature[z-BEM]{BEM}{Boundary Element Method}
\nomenclature[z-MPM]{MPM}{Material Point Method}
\nomenclature[z-LBM]{LBM}{Lattice Boltzmann Method}
\nomenclature[z-MRT]{MRT}{Multi-Relaxation 
Time}
\nomenclature[z-RVE]{RVE}{Representative Elemental Volume}
\nomenclature[z-GPU]{GPU}{Graphics Processing Unit}
\nomenclature[z-SH]{SH}{Savage Hutter}
\nomenclature[z-CFD]{CFD}{Computational Fluid Dynamics}
\nomenclature[z-LES]{LES}{Large Eddy Simulation}
\nomenclature[z-FLOP]{FLOP}{Floating Point Operations}
\nomenclature[z-ALU]{ALU}{Arithmetic Logic Unit}
\nomenclature[z-FPU]{FPU}{Floating Point Unit}
\nomenclature[z-SM]{SM}{Streaming Multiprocessors}
\nomenclature[z-PCI]{PCI}{Peripheral Component Interconnect}
\nomenclature[z-CK]{CK}{Carman - Kozeny}
\nomenclature[z-CD]{CD}{Contact Dynamics}
\nomenclature[z-DNS]{DNS}{Direct Numerical Simulation}
\nomenclature[z-EFG]{EFG}{Element-Free Galerkin}
\nomenclature[z-PIC]{PIC}{Particle-in-cell}
\nomenclature[z-USF]{USF}{Update Stress First}
\nomenclature[z-USL]{USL}{Update Stress Last}
\nomenclature[s-crit]{crit}{Critical state}
\nomenclature[z-DKT]{DKT}{Draft Kiss Tumble}
\nomenclature[z-PPC]{PPC}{Particles per cell}