\chapter{Thử nghiệm và đánh giá kết quả đạt được}

% **************************** Define Graphics Path **************************
\ifpdf
    \graphicspath{{Chapter3/Figs/Raster/}{Chapter3/Figs/PDF/}{Chapter3/Figs/}}
\else
    \graphicspath{{Chapter3/Figs/Vector/}{Chapter3/Figs/}}
\fi

\section{Thử nghiệm}
\subsection{Khoảng cách hoạt động}

\textbf{Phương pháp thực nghiệm:} Kết nối thiết bị Smart Keyring và thiết bị di động, liên tuc kiểm tra kết nối trao đổi tín hiệu giữa 2 thiết bị và tăng dần khoảng khoảng cách cho đến khi có báo hiệu mất kết nối. Thực nghiệm ở 2 trường hợp môi trường không vật cản và có vật cản.

Thử nghiệm được thực hiện với điện thoại Sony Xperia Z1:

\begin{table}[!ht]
	\centering
	\begin{tabular}{|c | c|}
		\hline 
		Các lần đo & Khoảng cách còn hoạt động  \\ 
		\hline
		Lần 1 & 22m \\
		
		Lần 2 &	22m \\
		
		Lần 3 &	21.5m \\
		
		Lần 4 &	23m \\
		
		Lần 5 &	23m \\
		
		Lần 6 &	22.5m \\
		
		Lần 7 &	22m \\
		
		Lần 8 &	21m \\
		
		Lần 9 &	22m \\
		
		Lần 10 & 22m \\		
		
		\hline 		
	\end{tabular}
	\caption{Thử nghiệm khoảng cách hoạt động khi không có vật cản}		
	\label{table:distance-no}
\end{table}

\begin{table}[!ht]
	\centering
	\begin{tabular}{|c | c|}
		\hline 
		 Các lần đo & Khoảng cách còn hoạt động  \\ 
		\hline
		Lần 1 &	8m \\
		
		Lần 2 &	8m \\
		
		Lần 3 &	8.5m \\
		
		Lần 4 &	8.5m \\
		
		Lần 5 &	8m \\
		
		Lần 6 &	7.5m \\
		
		Lần 7 &	8m \\
		
		Lần 8 &	7.5m \\
		
		Lần 9 &	8m \\
		
		Lần 10 & 8.5m \\		
		\hline 		
	\end{tabular}
\caption{Thử nghiệm khoảng cách hoạt động khi có vật cản}		
\label{table:distance}
\end{table}

Thực nghiệm cho thấy, phạm vi tối đa trung bình để việc truyền nhận dữ liệu còn chính xác là vào khoảng 22m đối với môi trường không vật cản, 8m đối với môi trường có vật cản( các 1 bức tường).

\newpage
\subsection{Năng lượng tiêu thụ}
Thông số tiêu thụ năng lượng của các linh kiện kiện theo nhà sản xuất:

• Tiêu thụ của HM-10: 0.2mA ở sleep mode, 8mA ở active mode

• Tiêu thụ của vi điều khiển ATmega328P: 0.2mA ở active mode

• Tiêu thụ của đèn LED: 

• Buzzer tín hiệu khi kích hoạt: 32mA

• Dung lượng nguồn sử dụng: Pin CR2032 dung lượng 220mAh

Thực tế năng lượng tiêu thụ:

\textbf{Phương pháp thực nghiệm:} kết nối thiết bị Smart Keyring với thiết bị di động, mỗi ngày kiểm tra kết nối 2 lần vào lúc 7h sáng và 5h chiều và tự đưa về chế độ Idle ở thời gian rảnh. Giữ thiết bị hoạt động liên tục với pin CR2032 cho đến khi hết pin.

\textbf{Kết quả: } Bắt đầu thực nghiệm vào lúc 7h sáng ngày 1/11, thiết bị không còn khả năng kết nối vào lúc 4h30 chiều ngày 18/11 mặc dù đèn nguồn vẫn còn sáng nhưng không đủ năng lượng để duy trì module BLE giữ kết nối.

\section{Đánh giá kết quả ứng dụng thực tế}
\label{result}
Phần này sẽ đánh giá về khả năng ứng dụng thực tế dựa trên kết quả thực nghiệm.

Về khoảng cách hoạt động tối đa và tính ứng dụng thực tế hiệu quả:

• \textbf{Môi trường không vật cản vào khoảng 22m: }hiệu quả trong việc báo mất kết nối trong các trường hợp thực tế như bỏ quên chìa khóa trên xe trong bãi giữ xe, thiết bị di động tại nơi công cộng vì khoảng cách hợp lý không quá ngắn và cũng như không quá xa.

• \textbf{Môi trường có vật cản vào khoảng 8m:} hiệu quả trong việc báo mất kết nối trong các trường hợp thực tế như để quên trong phòng, báo hiệu sớm để người dùng có thể biết ngay khi bỏ quên 1 trong 2 thiết bị khi vừa rời khỏi phòng. Và khoảng cách đủ để duy trì kết nối kích hoạt chế độ báo hiệu tìm kiếm khi thiết bị che khuất tầm nhìn giúp việc tìm kiếm nhanh chóng và hiệu quả hơn.

Về thời gian hoạt động: vì hạn chế trong việc không tối ưu thiết bị, đạt 18 ngày thay pin 1 lần là kết quả kém hiệu quả trong cuộc sống hằng ngày.